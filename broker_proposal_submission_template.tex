% Template v0.1: 02/06/2019. 
% Template for the LSST call for Letters of Intent for Community Alert Brokers
% The call for white papers can be found at ls.st/LDM-682


\documentclass[11pt, letterpaper]{article}
\usepackage{savetrees}
%\usepackage[top=1in, bottom=1in, left=1in, right=1in]{geometry}
\usepackage[utf8]{inputenc}
\usepackage{booktabs}
\usepackage{hyperref}


\title{Template for LSST Call for Proposals for Community Alert Brokers}
\author{A. Author (U. of Affiliation)\footnote{author@affil.edu}\ and B. Author (Affiliation Institute)}
\date{}

\begin{document}

\maketitle

\begin{abstract}
Please provide a short overview of your proposed broker here.

Please refer to LDM-723 for detailed submission instructions.
\end{abstract}


\section{Scientific goals of the proposed broker system}

\begin{itemize}
\item Describe the scientific aims of your proposed broker. 
\item Does it aim to add value to all LSST alerts, or a subset (e.g., supernovae or asteroids)?
\item What science questions will users of your broker be able to answer?
\end{itemize}

\section{Stream Access}

\begin{itemize}
\item Are you requesting direct access to the alert stream from LSST?  If not, how do you anticipate receiving alert data? 
\item Are you requesting the complete contents of all alerts, or a filtered subset of either the alerts (e.g., only alerts matching Solar System Objects) or their contents (e.g., alerts with postage stamps and history removed)?  Please be as specific as possible, as this will aid estimates of bandwidth usage.
\item If you are requesting a direct connection to the stream, what capacity do you have to forward alerts to more specialized broker systems downstream?  Are their other teams you have agreed to forward alerts to?
\end{itemize}

\section{Data Products \& Services}

\begin{itemize}
\item Describe the anticipated outputs and resultant data products of your processing of the LSST alert stream. 
\item What kinds of transformations, cross-matches, filters, and/or classifications will your broker perform?
\item How will users access them?
\item Who will be eligible to use your broker?
\item Do you anticipate multiple classes of users and/or differing resource allocations?
\item Describe any plans for services to enable follow-up observations or collaborations with follow-up facilities.
\end{itemize}



\section{Technical Implementation}

\begin{itemize}
\item Describe the software and hardware technologies with which you intend to implement your broker service, and discuss its architecture.
\item Discuss the maturity of the system---which parts are at a conceptual or early design phase?  Are some prototyped? Are any in operation already?
\item Describe the datacenter(s) in which you plan to host your service.
\end{itemize}


\section{Previous Results}

\begin{itemize}
\item Present examples of data products, services, and/or scientific results (if any) from your system on precursor data, which may include (but is not limited to) ZTF alerts, data from other time-domain surveys (e.g., \textit{Kepler}, \textit{TESS}, \textit{Gaia}, PanSTARRS, ATLAS, ASAS-SN, DECam, HSC...), simulations, or sample alerts processed by the LSST pipelines.
\end{itemize}
	
\section{Management Plan}

\begin{itemize}
\item Describe the number of FTEs required during the construction and operations phases.
\item Present a detailed timeline and major milestones for implementing, verifying, and validating the services to be developed.
\item Discuss potential or actual funding sources available to support these activities.	
\end{itemize}


\section{Proposing Team}

\begin{itemize}
\item Describe the roles and expertise of the proposers and any key personnel.
\item Describe related services the team has developed, if any.
\item Describe any experience the team has processing alert streams of currently operational astronomical surveys. 
\end{itemize}

\section{References}

\end{document}
